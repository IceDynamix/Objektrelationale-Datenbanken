\section{Vergleich}

\begin{frame}
    \frametitle{Relational vs. Objektrelational}
    \pause
    \begin{columns}[T,onlytextwidth]
        \column{0.5\textwidth}
            \textbf{Relational}
            \pause
            \begin{itemize}
                \item Tabellarisch (in Reihen und Spalten) dargestellt \pause
                \item Primärschlüssel \pause
                \item Beziehungen zu anderen relationalen Datenbanken \pause
                \item Grundlegende Datentypen \pause
                \begin{itemize}
                    \item int, float/double, byte \pause
                    \item string, char \pause
                    \item bool, null \pause
                \end{itemize}
            \end{itemize}

        \column{0.5\textwidth}
            \textbf{Objektrelational}
            \pause
            \begin{itemize}
                \item Auch tabellarisch dargestellt
                \item Alles was im Relationalen möglich ist \pause
                \item Kann zusätzlich komplexere Datentypen nutzen \pause
                \begin{itemize}
                    \item Arrays (int[], string[] etc. \dots) \pause
                    \item Objekte (selber definierte etc. \dots) \pause
                    \item Objekte (selber definierte etc. \dots) \pause
                \end{itemize}
            \end{itemize}
    \end{columns}
\end{frame}

\begin{frame}
    \frametitle{Objektorientiert vs. Objektrelational}
    \pause
    \begin{columns}[T,onlytextwidth]
        \column{0.5\textwidth}
            \textbf{Objektorientiert}
            \pause
            \begin{itemize}
                \item In (Instanzen von) Objekten dargestellt, nicht Tabellen \pause
                \item $\rightarrow$ hat ebenfalls komplexe Datentypen \pause
                \item Geeignet für eine Codebase, welche auf OOP basiert \pause
            \end{itemize}

        \column{0.5\textwidth}
            \textbf{Objektrelational}
            \pause
            \begin{itemize}
                \item Tabellarisch
                \item Ebenfalls komplexe Datentypen
            \end{itemize}
    \end{columns}
\end{frame}

\begin{frame}
    \frametitle{Ähnlichkeit}

    \begin{table}
        \begin{small}
            \label{tab:OODBvsORDB}
            \begin{center}
                \begin{tabular}[c]{|r|l|}
                    \hline
                    \multicolumn{1}{|c|}{\textbf{Objektorientiert}} &
                    \multicolumn{1}{c|}{\textbf{(Objekt)relational}} \\
                    \hline
                    Klassen                   & Tabellen \\
                    Instanzen (eines Objekts) & Reihen in einer Tabelle \\
                    Attribute                 & Spalten \\
                    Objektreferenzen          & Fremdschlüssel/Beziehungen \\
                    \hline
                \end{tabular}
            \end{center}
            \caption{Parallelen zwischen objektorientiert und objektrelational}
        \end{small}
    \end{table}

\end{frame}
