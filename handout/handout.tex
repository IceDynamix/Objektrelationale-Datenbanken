\documentclass[twocolumn]{article}

% PACKAGES --------------------------------------
\usepackage[utf8]{inputenc}
\usepackage[T1]{fontenc}

% GENERAL --------------------------------------

\usepackage{tikz} % Drawings
\usepackage{wrapfig} % Inline images
\usepackage{amsmath,amssymb,amsthm} % Advanced math
\usepackage{xcolor} % Colored text
\usepackage{hyperref} % Clickable TOC
\renewcommand{\baselinestretch}{1.2} % Line height
\usepackage{graphicx} % Graphics *but better*
\usepackage{adjustbox}
\usepackage{bookmark}
\usepackage{minted}
\usepackage{titling}

\setlength{\droptitle}{-8em}     % Eliminate the default vertical space
\addtolength{\droptitle}{-4pt}   % Only a guess. Use this for adjustment

% \graphicspath{{./img/}}

% \input{lib/glossary.tex} % Glossary
% \input{lib/nomenclature.tex} % Nomenclature/Symbolverzeichnis

% MACROS ----------------------------------------

% Displaystyle fractions
\newcommand\ddfrac[2]{\ensuremath{\frac{\displaystyle #1}{\displaystyle #2}}}

% TIKZ ------------------------------------
\usepackage{pgf,tikz,pgfplots}
\usepackage{mathrsfs}
\usetikzlibrary[patterns]
\pgfplotsset{compat=1.15}
\usetikzlibrary{arrows}
\newcommand{\degre}{\ensuremath{^\circ}}

% GERMAN --------------------------------------

\usepackage[ngerman]{babel} % german
\usepackage{csquotes}

% https://www.overleaf.com/learn/latex/Typesetting_quotations
% use \say{} to quote something
\usepackage[
    left = \glqq{},
    right = \grqq{},
    leftsub = \glq{},
    rightsub = \grq{}
]{dirtytalk} % Correct quotation marks

\newtheorem*{theorem}{Hypothese} % Rename
\addto\captionsngerman{%
    \renewcommand{\abstractname}{Vorwort}
}

% BIBLIOGRAPHY --------------------------------------

\usepackage[backend=biber,sorting=none]{biblatex}
\addbibresource{../bib/bib.bib}

% METADATA --------------------------------------

% SETUP --------------------------------------

\title{Objektrelationale Datenbanken - Handout}
\author{******** Nguyen}
\date{\today}



\begin{document}
    \maketitle

    \section{Definition}
    Eine objektrelationale Datenbank ist eine Datenbank, welches auf einer relationalen Datenbank aufbaut, aber Elemente aus dem objektorientiertem Programmieren bzw. einer objektorientierter Datenbank einbindet. Dabei stehen Features, wie
    komplexe Datentypen und Vererbung, im Vordergrund. Man könnte sagen, dass eine objektrelationale Datenbank von einer relationalen erbt. Insgesamt bilden folgende Konzepte den Grundbaustein für eine objektrelationale Datenbank:

    \begin{itemize}
        \item Aus einer relationalen Datenbank
        \begin{itemize}
            \item Tabellenartige Darstellung mit Reihen und Spalten
            \item Primärschlüssel
            \item Fremdschlüssel/Beziehungen zu anderen Tabellen
        \end{itemize}
        \item Aus einer objektorientierten Datenbank
        \begin{itemize}
            \item Komplexe Datentypen, wie Arrays oder Objekte
            \item Vererbung
        \end{itemize}
    \end{itemize}

    \section{Beispiele}
    Eine relationale Datenbank kann keine komplexen Datentypen, wie Arrays oder Objekte einbinden. Eine int[] Spalte wäre also undenkbar. Aber ebenso auch eine Spalte mit selbst-definierten Objekten. Für beide müsste man separate Tabellen anlegen. Folgendes wäre zum Beispiel in einer objektrelationalen Datenbank möglich:

    \begin{minted}{java}
        public class Image {
            public String path;
            public Date dateCreated;
            public Date dateModified;
            // [sizeX, sizeY]
            public int[] dimensions;
            public int fileSize;
        }
    \end{minted}

    \begin{table}[!htb]
        \centering
        \begin{adjustbox}{width=0.45\textwidth}
            \small
            \begin{tabular}[c]{|c|c|c|c|c|c}

                \hline

                \multicolumn{1}{|c|}{\textbf{id}} &
                \multicolumn{1}{c|}{\textbf{name}} &
                \multicolumn{1}{c|}{\textbf{vorname}} &
                \multicolumn{1}{c|}{\textbf{alter}} &
                \multicolumn{1}{c|}{\textbf{portrait}} &
                \multicolumn{1}{c}{\dots} \\

                \multicolumn{1}{|c|}{\textit{int}} &
                \multicolumn{1}{c|}{\textit{String}} &
                \multicolumn{1}{c|}{\textit{String}} &
                \multicolumn{1}{c|}{\textit{int}} &
                \multicolumn{1}{c|}{\textbf{Image}} &
                \multicolumn{1}{c}{\dots} \\

                \hline

                0  & Mustermann  & Max    & 27             & <Instanz>      & \dots \\
                1  & Mustermann  & Marie  & 26             & <Instanz>      & \dots \\
                2  & Fröhlich    & Nico   & 18             & \textit{null}  & \dots \\
                3  & Hammer      & Niko   & \textit{null}  & <Instanz>     & \dots \\
                $\vdots$ & $\vdots$ & $\vdots$ & $\vdots$ & $\vdots$ &

            \end{tabular}
        \end{adjustbox}
        \caption{Tabelle mit Objekten}
        \label{tab:erwachseneMitBildern}
    \end{table}

    \section{Quellen}
    \nocite{*}
    \printbibliography

\end{document}
